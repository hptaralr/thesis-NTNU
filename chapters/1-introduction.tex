\chapter{Introduction}
\label{chap:Intro}

\section{Background}
\label{section:background}
Mobai is a spin off company from the research developed in the Norwegian Biometrics Laboratory at the Norwegian University of Science and Technology. They provide solutions for facial recognition, biometrical attack detection and face mor-ph detection. To create models for detection of biometrical attributes, artificial intelligence (\acrshort{ai}) and machine learning are essential tools for Mobai. In addition to the models, having access to appropriate datasets plays a crucial role in training and developing new models. An important focus of Mobai is using different \acrlong{fiqm}s (\acrshort{fiqm}s) to determine the quality of facial images in a dataset. In order to train models, quickly assess several datasets or create new datasets, it is important to know the quality of facial images. Therefore, Mobai now aims to develop an application that automates this process. 

\section{Subject Area}
Digital image processing is a rapidly growing field within the world of engineering and computer science. A great amount of research has been done in this field of study, paving the way for multimedia systems to become one of the pillars of the modern information society. Digital image processing is used in a variety of technologies, including face detection and face recognition which itself could be categorized under the broader field of pattern recognition. Digitalization has drastically changed peoples everyday lives over the last decade, and with that change, biometrics has become more relevant and important than ever. 

Solutions such as face recognition, presentation attack detection and face morph detection all heavily rely on the quality of the facial images used for machine learning training. The quality of facial images are dependent on several factors which FIQMs have learned trough artificial intelligence and machine learning. The performance of FIQMs can be measured by comparing the quality scores with human assessment. This thesis is mainly focused on Face Image Quality Assessment (\acrshort{fiqa}) which plays a key role in improving the accuracy of face recognition systems. 

\section{Task Description}
\label{sec:TaskD}
This bachelor project can be divided into two main parts, a programming part (mainly Chapter \ref{chap:objective}) and a research part (mainly Chapter \ref{chap:subjective}).

\subsection*{Programming}
The programming part involves the creation of a web application that uses two FIQMs provided by Mobai for evaluating the quality of facial images. The application will create a report which contains the FIQMs´ calculations on a set of images. The key functionalities of the application expected from Mobai are: 
\begin{itemize}
    \item The user should be able to read/select images from a local machine.
    \item The user should be able to upload images to a directory.
    \item The user should be able to execute the two FIQMs on the uploaded images.
    \item The user should be able to display the results from the FIQMs. 
\end{itemize}

\subsection*{Research}
The research part consists of conducting a subjective experiment. The subjective experiment involves collecting ground truth data by having subjects evaluate the quality of facial images from a dataset based on certain criteria. During the experiment, observers will be shown different facial images of varying quality and asked to label the images into predefined categories. The subjective results will then be used to compare with the objective results from the FIQMs. By comparing the subjective results from the experiment with the objective measure calculated by FIQMs, we are then able to evaluate the performance of FIQMs. 

\section{Delimitation}
\label{sec:delimit}
In this project, our task was not solely made up of pure programming. Fortunately, the subject field we were working within allowed research which we found to be a great experience in the final step of our bachelor studies. As an example, facial images used for research in the field of face quality assessment and face recognition were usually taken from a straight angle with little to no tilting of the camera lens. After a literature review it was evident that studies addressing camera tilting was very limited. With the ongoing pandemic, wearing face masks in public has become a habit. Even though wearing a mask is a new aspect of our daily lives, the performance of FIQMs on images which show a subject wearing a face mask has not yet been studied. In that case, we wanted to assess the performance of the FIQMs on face mask images. For the mentioned reasons, the bachelor group proposed the collection of a new dataset which Mobai supported the initiative. The mentioned dataset will not only be used by Mobai in their studies but can also be used by the bachelor group. Further information about the collected dataset is provided in Section \ref{sec:ownData}. Finally we should point out that with the exception of our dataset, all FIQMs and datasets were provided to the bachelor group by Mobai. 

\section{Target Groups}
\label{sec:TargetGroups}
In general, this project is targeted towards two groups, Mobai and the readers of the thesis. 

\subsection{Users of the Web Application}
The web application will be used by employees at Mobai to evaluate the performance of the FIQMs on different datasets. Based on the Non Disclosure Agreement (NDA) signed by the bachelor group (Appendix \ref{app:project-agreements}), because of possible competitive companies neither the source code or the application should be distributed to others except Mobai.

\subsection{Thesis Readers}
The target group for the bachelor thesis are people who need insight in how we did the project from a developer point of view as well as a scientific point of view. This includes but is not limited to our thesis supervisor, client and fellow students with similar background that can be interested in reading the thesis.

\section{The Team}
In this section we will introduce the members in our team. Here we look at their interests, responsibilities and their academic backgrounds. 

\subsection{The Members}

\subsubsection*{Hans Petter}
Hans Petter is a computer engineer who is interested in mathematics and artificial intelligence. He is interested in Python programming and was therefore one of the main developers of the backend. Hans Petter had the main responsibility for implementing \acrshort{api}s to the frontend. He also helped out the other team members whenever there were any API errors connecting Flask to React or general Python problems. In addition to the application task, Hans Petter collected the data from the subjective experiment to make it workable. 

\subsubsection*{Walid}
Walid has prior knowledge with artificial intelligence and finds topics that mix statistics and mathematics with artificial intelligence interesting. Throughout the project, Walid was mainly involved in the creation of the backend logic with Hans Petter. In particular, he developed API endpoints used by the frontend and was responsible for processing the objective and subjective data.   

\subsection*{Julian}
Julian is a computer engineer with an interest in programming, mathematics and artificial intelligence. His background consists of different subjects regarding programming, mathematics, software security and application development. Julian's main responsibility was creating the frontend of the application and handling the data received by the backend. 

\subsubsection*{Kjetil}
Kjetil is a developer with an interest in frontend web development, JavaScript and Python programming. Due to his background and interests, his main responsibility was to work with the frontend and help with the backend development. 


\subsection{Academic Background}
\label{section:academic background}
All group members are studying for a bachelor´s degree in computer engineering, hence our academic backgrounds are closely similar. However, during the fifth semester, we all chose different subjects. Hans Petter and Walid both chose \textit{artificial intelligence}, \textit{software design} and \textit{calculus 3}. Julian had the elective subjects \textit{application development}, \textit{software security} and \textit{calculus 3}, while Kjetil studied the subjects: \textit{ergonomics in digital media}, \textit{infrastructure as code} and \textit{application development}. 

Throughout our bachelor´s program we have acquired a solid foundation when it comes to the software development process and scientific thinking by finishing courses like \textit{software engineering}, \textit{algorithmic methods}, \textit{operating systems}, \textit{scientific computing}, \textit{calculus} and \textit{physics}. The compulsory courses we finished helped us see different approaches to the same problem. There will always be several ways to approach a task and a bachelor´s degree in computer engineering has cemented that statement. Our prior knowledge made it easier to understand and process the main concepts of the new topics we had to learn. 
During this semester, we have acquired more knowledge and built on our foundation. We had to perform research on specific topics in backend and frontend programming, as these were topics we were unfamiliar with. More specifically, the part of connecting the frontend to the backend by using API calls, was something we had never done before. In addition to that, a great amount of time was also spent on learning how to conduct and perform scientific experiments in a way that satisfies both \acrfull{gdpr} and the relevant \acrshort{iso}-standards.  

\section{Why Did We Choose This Task?}
Although stationed in Gjøvik the group had never heard of Mobai or what they do. Their task description seemed interesting, and we were early to contact them. Already at the first meeting, before any bachelor tasks were selected, we got a very positive impression. The participants from Mobai were very curious and enthusiastic. They had clear ideas and suggestions on how we could approach the task. After this meeting, the bachelor thesis choice became an easy decision. 

A reason why we chose this task was the field of work. All group members were interested in machine learning and artificial intelligence. Therefore, we saw the topic as a perfect opportunity to have a hands on experience working in this field. Another reason for our choice was the width of the project. We felt that we could use experiences and knowledge learned throughout the bachelor courses in this work. We had to combine math, statistics and programming with research, which given our bachelor's program suited us well. This kind of scope was complex, but manageable, and challenged us in several ways in terms of coding and research. We were motivated by the fact that Mobai expected us to create an application that they would utilize in addition to creating a subjective experiment.   

\subsection{Roles}
\label{subsec:roles}
Our project work areas were mainly divided into two categories: research and programming. While studying FIQMs, the subjective assessment and the creation of a subjective experiment which can be seen as research were done collectively to ensure equal professional foundation. When it came to the programming part, each group member were delegated responsibility more individually. However, due to the nature of the work, we had to closely collaborate with each other with backend and frontend because its dependency. During the project, the following roles were assigned to each group member:
\begin{itemize}
    \item Kjetil Grosberghaugen was scrum master and developer. His main responsibility was frontend, written in React.js. As a scrum master he ensured the development process flew evenly and led sprint meetings.
    \item Julian Nyland Skattum was a developer with main responsibility in frontend. He was also responsible for programming language choices as well as the overall application appearance. 
    \item Hans Petter Fauchald Taralrud was a developer with main responsibility\\ in backend. 
    \item Walid Demloj was a developer with main responsibility in backend. 
\end{itemize}

Every member were participating in the full stack application despite their area of responsibility. The reason for having individual responsibilities was to easily maintain that the requirements were met. The group members had joint responsibility for the project plan (see Appendix \ref{app:project-plan}) and thesis as well as completing tasks according to the created requirements. 

Seyed Ali Amirshahi was our supervisor. The project group had scheduled meetings every Monday at 12:00. Ali's role in the project was to guide us throughout the project. He would evaluate the project plan, thesis and answer professional questions. Summaries of the meetings with Ali can be found in Appendix \ref{appendix:supervisor}.

Kjartan Mikkelsen started as the product owner, but after about three months in the process, he left the company. Our new product owner and contact was Guoqiang Li. He expressed Mobai's vision and wishes during the project. We also got technical assistance from other employees at Mobai. Summaries of the meetings with Mobai can be found in Appendix \ref{appendix:Mobai}.

\subsection{Decision Making}
The group decided to go for a collaborative decision-making \cite{GroupDecisionMaking} approach when making choices throughout the project. Decisions like frontend and backend environment and experiment platform were decided collectively with knowledge and experience in mind. A touch of prior knowledge benefited our project, so we did not need to learn everything from scratch. Making choices in groups led to a comfortable team atmosphere with a good relationship between each team member. The decentralization of decision responsibilities made it easier to contribute in the decision making, because possible failures would be shared in the group. However, if a discussion had reached an impasse, the group leader had the final say. Choices related to project work were brought up in the daily scrum meetings, as they offered frequent changes as well as a cordial environment to give feedback and improvements. Choices related to project organization were arranged in the sprint retrospective meetings, every other week. 

\section{Thesis Structure}
It is important to emphasize that this thesis is atypical. As almost every bachelor task include reporting the development process, our thesis adds an extra dimension in terms of research. The subjective part requires thorough research to be able to correctly compare with the results provided by the application. All this is reflected in the way we structured our thesis. 
We have chosen to structure the thesis as follows: in Chapter \ref{chap:Specification}, we are discussing our choice of development methodology. We look at pros and cons for the methodology, address other possible development methods and elaborate the current development layout. Following the chosen development approach is an in-depth risk analysis. Chapter \ref{chap:FQA} is an important part of the thesis. Here we define some of the most essential concepts used in the project. We describe the definition of a good facial image in face recognition as well as introducing the FIQMs used in the application. Next, in Chapter \ref{chap:objective} we will look at how the face image metric application has been implemented and the reasoning behind our choices. In Chapter \ref{chap:subjective} we go through the process of conducting a subjective experiment to gather ground truth data. We also introduce the datasets used in the experiment and explain our own dataset creation. 
Both the objective assessment and subjective experiments are merged together in Chapter \ref{chap:Results}, where their results are compared. Following the results, we can conclude the FIQMs strengths and weaknesses based on their correlation with the subjective scores. In the conclusion Chapter \ref{chap:Conclusion} we conclude the results and introduce further work based on our research. The reasoning behind this project structure is mainly to provide the thesis readers a clear overview and a sequential story. Since the project's contents of the work are split, it is meaningful to describe the workload in separate chapters. 
\chapter{Specification}
\label{chap:Specification}
In this chapter we will start by looking at our chosen development method for the project and obstacles that arose with it. To end the chapter of, an in depth risk analysis will be presented.


\section{Development method}
Projects of this size are dependant on well structured development methods to succeed as it affects the way teams are organized and tasks are delegated. The prioritization of workload is also affected with the choice of development method. 
This project was characterised by uncertainty, a limited amount of labor and a development phase with requirements that may change. These characteristics paved the way for us to pick an agile software development method which was suggested by our client Mobai.  
\begin{enumerate}
    \item \textbf{Constant feedback} 
    
    \hspace{0,5cm}Our group consists of four third year students with little to none experience with comprehensive projects of this size. Such lack of experience alone causes some uncertainty in regards to deadlines, survey work and development. Having regular meetings with both Mobai and our project supervisor as well as receiving regular feedback, will decrease the chance to drift of track. This also ensures that our final product is as close to Mobai's vision as possible. Agile software development methods provides the client with the chance to be far more involved in the project. This aspect itself was an important point to take into consideration.
        
    \item \textbf{Small team}
    
    \hspace{0,5cm}For a team like us, an agile method is more suited because small teams are better positioned to efficiently and effectively manage events like meetings \cite{SmallTeams}. The reduction of communication channels in a team decreases the possibility of misinterpretations among members. A small team is more likely to have an attentive communication where decisions are made quickly.
    \newpage
    
    \item \textbf{All known requirements are not final}
    
    \hspace{0,5cm}In agile development methods it is expected that all the requirements are not finalized or even known in advance. Throughout the development process new requirements may be added and existing ones can be refined to better suit the upcoming application. This agility is not achieved by plan driven approaches because all requirements must be finalized in the planning phase. Bachelor projects rarely turn out exactly as planned, which is why an agile development method that handles changes would be beneficial. 
\end{enumerate}

\subsection{Considered models}
Although the scale tipped towards an agile methodology, we also included plan-driven approaches in our considerations in case they were a better solution. Following is our reasoning behind considered models.  

\subsubsection*{Waterfall method}
The fundamental process activities is broken down into linear sequential pha-ses\cite{WaterfallMethod}. Each phase has to be completed before it cascades into the next. This allows for departmentalization and control where deadlines can be set for every phase, resulting in a clear workflow. However, this model would not be suitable for our development due to its lack of revision. Should the development phase run into issues that was not taken into consideration in the planning or design phase, the design phase would have to start over. This would be very time consuming. Also, not having a working product until late in the development life cycle would be risky in regards to our programming uncertainty.   

\subsubsection*{Extreme Programming}
Extreme Programming (XP) emphasises five key values: communication, feedback, simplicity, courage and respect. It strives for high team efficiency and higher quality code by implementing 12 practises such as, refactoring, pair programming, test first development, continuous integration and simple design. Programmers are expected to work closely together, which is encouraged by the values mentioned \cite{eXtremeP}. The main downside however, is its strict and difficult way of working. It requires great self discipline to follow all the practises the method is built upon. XP mainly focuses on how software development should be approached, and since our project was not solely based on development, the model did not seem to fit our needs.  
\newpage

\subsubsection*{Kanban}
Kanban provides solid structuring and flexibility. The methodology is all about limiting work in progress as well as maximizing the work efficiency. It is based on a continuous workflow structure with continuous deliveries with no set dates. This ensures the team is ready to adapt to change whenever priorities may change. The kanban team are reliant on each other to succeed. No predefined roles are set, which means it is a collective responsibility of the team to work together and finish tasks. Likewise, no set dates are placed for when certain functionality should be released \cite{kanban}. We felt that this ``go with the flow'' mentality of kanban would lead to more uncertainty regarding the bachelor groups progress. The roles and responsibilities structuring of kanban, reinforced this mindset.  

\subsubsection*{Scrum}
Scrum is adjustable, flexible and a fast agile method that largely involves the client throughout the project cycle\cite{ScrumDefinition}. The requirements are never entirely determined which allows for updating the product. Small teams with defined roles are appropriate to provide different responsibility within the team. Having regular meetings ensures good communication as well as keeping team members up to date with the continuous progress.    

\subsection{Our model}
\label{sec:OurModel}
We chose to use a scrum model for this project. In addition, we wanted to implement the pair programming practise from XP since it would help to produce high-quality code as well as share experience in the field of programming. The pair programming was executed by screen sharing or with Visual Studio Code live-share. We also wanted to easily manage our working tasks. Therefore, we applied a kanban board to get a clear overview of the tasks that needed to be done. Trello \cite{Trello} was used for tracking the tasks and the tasks were divided into three phases: To-Do, Doing and Done. Since Kanban was not our main method, no specific rules or limitations were set considering the amount of tasks in each phase. Due to the project's nature, the adaptability and flexibility provided by scrum seemed to be a safe option which could also increase the chances of success in the project. The way the model organizes workload into different sprints and facilitates to pre-defined member roles helped us to keep up with the progression made throughout the project. Should we finish Mobai's desired application quickly it was suggested by Mobai that additional functionality could be added to the solution if our time schedule allowed for it. Changes in requirements were something Mobai were open to discuss and scrum handles this well, which was another reason for our choice. 

\subsection{Scrum layout}
We had sprints with a two-week duration. This is because, one-week sprints felt short and would lead to excessive meetings which would affect our time management. Longer sprints however, tend to involve the client to a lesser degree, which would not be ideal in our case. The involvement of the client was not only to have continues feedback to ensure that we satisfied their requirements, but it was what Mobai had emphasized on at the start of the project. The sprints consisted of several meetings and scrum practises, such as Sprint planning meetings, Daily scrums, Sprint review meetings and Sprint retrospective meetings. These meetings were all followed to a certain extent. 

\subsubsection*{Sprint planning meeting}
The start of the sprints took place on Tuesdays, (with the first meeting on the 2\textsuperscript{nd} of February) and were expected to finish by 11:00 every other Monday. The weekly scheduled meetings with our bachelor supervisor took place every Monday at 12:00, which gave us at least one hour to prepare both the next sprint and what we should discuss during the meeting. This is when all our sprint planning meetings were held and conversations about what features should be included in the upcoming sprint were planned. \textit{Planning poker} was used to estimate the time spent on different tasks.

\subsubsection*{Daily scrum}
These short meetings were held at the start of every day at 9:00 and were included to assure the members were kept up to date about the progress of the project.

\subsubsection*{Sprint review meeting}
The sprint review meetings with Mobai were arranged every other Tuesday, at the end of each sprint. Mobai was informed about our progress and provided us feedback. While these meetings were mainly attended by the product owner (Dr. Guoqiang Li) who provided feedback from the client's side, other employees of Mobai also attended these meetings. 

\subsubsection*{Sprint retrospective meeting}
At the end of each sprint right after our sprint review meetings with Mobai, time was delegated into the sprint retrospective meetings. During these meetings we reflected about what had went well and tried to improve these aspects for the next sprints. A crucial part of these meetings was to decide if our time estimates were on point. 

\subsubsection{Definition of done}
We used the scrum-pattern Definition of done \cite{DoneDone} to collectively define what development-tasks were considered done. In partnership with Mobai we came to an agreement upon different criteria that form the definition of done. It was important to set different criteria to different work tasks (there should be a difference between coding-criteria and report-criteria). This pattern gave us a familiar understanding of work quality and absoluteness. We also obtained good habits in our workflow using the definition of done as a checklist to correlate with the user stories. In that way we prevented possible delays occurring in the development process. Here is an example of our report-criteria: 

\begin{enumerate}
    \item The section is completed according to the member.
    \item The member has analyzed the section's contents.
    \item The section is checked for typos.
    \item The whole group has read and approved the section.
\end{enumerate}

\subsection{Difficulties with our model}
At the early stages of planning our bachelor project, our own development method seemed to fit all our requirements and needs with few disadvantages. However, once the development began, we quickly realised specific elements that were not desirable. 

Our pair programming practise inspired by XP did not work out as we intended or predicted it would. As it in the initial start of the project was very helpful to share experience, later in the development process, we came to the realization that it was very time consuming and limiting. Originally, the pair programming was supposed to be done on Visual Studio Code's Live Share, which allows developers to collaborate in real-time. However, the Live Share platform was constantly lagging for some of the members, which was why we decided to achieve pair programming by screen-sharing instead. This method forced one member to write code while the other observed and gave input which we found out to be unproductive and a waste of our limited resources. Because of that, we ended up coding the product mostly individually. The quality of the code was likely affected by this, however the productivity of the team became increasingly higher. The progress was easily traceable given our everyday meetings on online platforms.   

All meetings associated with scrum presented in the end of the previous section were beyond doubt time consuming and required certain planning in advance. The planning poker time estimating game was abandoned after a while because we felt it took valuable time away from the project. At the start it was a great tool that served us quite well, but as the project became clearer, we developed an understanding of time estimates. Because of that, we ended up estimating tasks verbally, which saved us time and resources.  

\section{Risk analysis} 
Almost every decision made throughout the project involved a risk. A risk was made up by the likelihood of something going wrong and the negative conse-quences entailed with that risk. Therefore, weighing up the risks before making a decision was very important. The better we understood occurring risks, the more we were prepared to manage them. Carrying out a risk analysis was beneficial for us to make sure every decision were robust and well considered. Our risk identification was done by discussing possible events that could occur, as well as identifying their likelihood and consequences. Prior knowledge from earlier projects helped us estimate whether a risk was likely or unlikely to happen. Since we were working agile, not all risks were decided in advance, but some were appended along the project process. The team categorized the risks according to the type of risk. The risk categories we used were: Project, Product and Business, which had been discussed in  detail \cite{RiskAnalysis}. 

\begin{itemize}
    \item \textbf{Project risks} 
        \par \hspace{0,5cm} Risks that affect available resources assigned to the project. These risks interfere with the schedule and may hinder reaching planned deadlines. 
    \item \textbf{Product risks} 
        \par \hspace{0,5cm} Risks that affect the overall performance of the system created. Risks in this category may affect the functionality which can reduce the overall quality. 
    \item \textbf{Business risks}
        \par \hspace{0,5cm} These risks affect the organization developing of a product. During this project, such risks will mainly be related to the team members as well as collaborators such as Mobai and our supervisor. 
\end{itemize}

Each risk listed below in the section, was evaluated and assessed based on its consequence and likelihood (Table\ref{table:colorC}). The numbers inside the Table reference what the risks were, and have nothing to do with the priority or severity. 
    
\begin{itemize}
    \item \textbf{Risk 1:} Group members leaving the project.
    \item \textbf{Risk 2:} Sprint deadlines are not met.
    \item \textbf{Risk 3:} Bachelor thesis is not finished in time.
    \item \textbf{Risk 4:} Mobai´s expected involvement turn out to be minimal. 
    \item \textbf{Risk 5:} A similar product launches.
    \item \textbf{Risk 6:} The software does not fulfill the minimum requirements.
    \item \textbf{Risk 7:} Inadequate planning and execution of subjective experiment.
    \item \textbf{Risk 8:} Loss of vital documents or source code.
    \item \textbf{Risk 9:} Sickness among group members, project supervisor or employees at Mobai.
    \item \textbf{Risk 10:} Application breaking bugs.
    \item \textbf{Risk 11:} Planned tools for development of the survey and application prove to be unsatisfactory. 
    \item \textbf{Risk 12:} Covid-19 restrictions interfere with the subjective experiment. 
    \item \textbf{Risk 13:} Unable of connecting the frontend to the backend logic. 
    \item \textbf{Risk 14:} Unrealistic demands are made along the way by the Product Ow-ner. 
    \item \textbf{Risk 15:} Additional help is required by other group members to overcome a specific problem.
\end{itemize}

\begin{table}[h]
\caption{Consequence and likelihood color coding risk matrix}
\label{table:colorC}
\resizebox{\textwidth}{!}{%
\begin{tabular}{ll|l|l|l|l|}
\cline{3-6}
 &
  \textbf{} &
  \multicolumn{4}{c|}{\textbf{Consequence}} \\ \cline{3-6} 
 &
  \multicolumn{1}{c|}{} &
  \cellcolor[HTML]{C0C0C0}{\color[HTML]{333333} Minor} &
  \cellcolor[HTML]{C0C0C0}{\color[HTML]{333333} Moderate} &
  \cellcolor[HTML]{C0C0C0}{\color[HTML]{333333} Major} &
  \cellcolor[HTML]{C0C0C0}{\color[HTML]{333333} Critical} \\ \hline
\multicolumn{1}{|l|}{} &
  \cellcolor[HTML]{C0C0C0}Unlikely &
  \cellcolor[HTML]{32CB00} R5 &
  \cellcolor[HTML]{32CB00} &
  \cellcolor[HTML]{F8FF00} R1 \& R4 \& R6 &
  \cellcolor[HTML]{FE0000} R3 \& R8 \& R13 \\ \cline{2-6} 
\multicolumn{1}{|l|}{} &
  \cellcolor[HTML]{C0C0C0}Likely &
  \cellcolor[HTML]{32CB00} R14 &
  \cellcolor[HTML]{F8FF00} R11 &
  \cellcolor[HTML]{F8FF00} R2 &
  \cellcolor[HTML]{FE0000} R10 \\ \cline{2-6} 
\multicolumn{1}{|l|}{} &
  \cellcolor[HTML]{C0C0C0}Quite likely &
  \cellcolor[HTML]{F8FF00} R15 &
  \cellcolor[HTML]{F8FF00} &
  \cellcolor[HTML]{FE0000} R7 \& R9 {\color[HTML]{FFFFFF} } &
  \cellcolor[HTML]{FE0000} \\ \cline{2-6} 
\multicolumn{1}{|l|}{\multirow{-4}{*}{\rotatebox[origin=c]{90}{\textbf{Likelihood}}}} &
  \cellcolor[HTML]{C0C0C0}Very likely &
  \cellcolor[HTML]{F8FF00} &
  \cellcolor[HTML]{FE0000} R12 &
  \cellcolor[HTML]{FE0000}{\color[HTML]{FFFFFF} } &
  \cellcolor[HTML]{FE0000} \\ \hline
\end{tabular}%
}
\end{table}

Table \ref{table:colorC} showcases a standard consequence/likelihood risk ranking matrix split into the likelihood of a risk occurring, and its corresponding consequence (i.e severity). The consequences ranges from minor to critical and the likelihood from unlikely to very likely. The three colors represent different type of risk levels. The green color indicates low risk, yellow indicated moderate risk and red represent risks of great importance. 

We chose a $4\times4$ matrix with three color codes because it covers and classifies a wide specter of risks. We could have settled for a $3\times3$ matrix, but we felt that risks would overlap and therefore a more severe risk could be mixed with a risk with lesser severity. Although a $5\times5$ matrix could have also be used, but since our project was limited in size relative to corporate projects, we felt that this type of granularity was not needed. 

In the next pages, all the above mentioned risks are placed into the three categories presented, and discussed with countermeasures. 

\begin{table}[h]
\caption{Project risks}
\resizebox{\textwidth}{!}{%
\begin{tabular}{|l|p{4cm}|p{7cm}|}
\hline
\rowcolor[HTML]{C0C0C0} 
\textbf{ID}                 & \textbf{Description} & \textbf{Countermeasures} \\ \hline
\cellcolor[HTML]{FE0000}R3  &    
Bachelor thesis is not finished in time. 
&
Base our time estimates on worst-case scenarios, that way if tasks are finished before a set date, some extra time can be added to the end for polishing purposes.
\\ \hline
\cellcolor[HTML]{FE0000}R7  &    
Inadequate planning and execution of subjective experiment. 
& 
Creating the survey at an early stage and studying the pitfalls in conducting surveys will minimize the risk of designing a lackluster and confusing survey. For more in depth and specific countermeasures, see Section \ref{sec:ConductingSurvey} Conducting the survey. 
\\ \hline
\cellcolor[HTML]{FE0000}R8  &      
Loss of vital documents or source code. 
& 
Using backup-technologies when writing the report and coding the software prevents loss of important content. The thesis will be synced with a GitHub repository and changes to the Overleaf project will regularly be pushed to the repository. 
\\ \hline
\cellcolor[HTML]{FE0000}R12 &      
Covid-19 restrictions interfere with the subjective experiment. &
It is highly likely that we won´t be able to conduct the experiment in a controlled environment which may affect subjects image evaluations. An instruction manual will be provided to all participants. 
\\ \hline
\cellcolor[HTML]{F8FF00}R2  &     
Sprint deadlines are not met. 
&      
Having a consistent workflow and a regular meeting schedule, makes it easier to meet the deadlines. If there are too many tasks in the upcoming sprint, a rescheduling may occur and the Product Owner will be informed. 
\\ \hline
\cellcolor[HTML]{F8FF00}R4  &      
Mobai´s expected involvement turn out to be minimal. 
&
Doing a lot of early research about our subject field is important to understand the basics, which we can build on ourselves. In addition, our supervisor should be utilized as best as possible.
\\ \hline
\cellcolor[HTML]{F8FF00}R15 &        
Additional help is required by other group members to overcome a specific problem. &
Our research and planning should make us capable of knowing a little about several topics. Members should therefore be able to assist each other if needed.
\\ \hline
\end{tabular}%
}
\end{table}


\begin{table}[h]
\caption{Product risks}
\resizebox{\textwidth}{!}{%
\begin{tabular}{|l|p{4cm}|p{7cm}|}
\hline
\rowcolor[HTML]{C0C0C0} 
\textbf{ID}                 & \textbf{Description} & \textbf{Countermeasures} \\ \hline
\cellcolor[HTML]{FE0000}R10 &
Application breaking bugs. 
& 
Analyze what critical bugs can occur when developing the product at an early stage. Applying continuous testing throughout the development phase should mitigate large faults in the application. 
\\ \hline
\cellcolor[HTML]{FE0000}R13 &
We are unable to connect the frontend to the backend logic.  
&
We should minimize the use of hard coding that may interfere with connecting React to Flask. Additionally, study materials which could help us in this issue should be studies by the members. Lastly, Mobai´s development team may assist us if needed. 
\\ \hline
\cellcolor[HTML]{F8FF00}R6  &
The software does not fulfill the minimal requirements. 
&  
Keeping a consistent workflow insures maximum efficiency. Should problems, which forces us to exceed deadlines by a great margin occur, a meeting with Mobai will be held where the requirements should be reconsidered. 
\\ \hline
\cellcolor[HTML]{F8FF00}R11 &
Planned tools for development of the survey and application prove to be unsatisfactory. 
& 
All tools should be replaceable and carefully considered through research to minimize the use of inefficient tools. Most of our requirements should be met by the tool. If several tools match our needs, the most suitable and quickest to learn will be chosen.  
\\ \hline
\end{tabular}%
}
\end{table}

\begin{table}[h]
\caption{Business risks}
\resizebox{\textwidth}{!}{%
\begin{tabular}{|l|p{4cm}|p{7cm}|}
\hline
\rowcolor[HTML]{C0C0C0} 
\textbf{ID}                 & \textbf{Description} & \textbf{Countermeasures} \\ \hline 
\cellcolor[HTML]{FE0000}R9  &
Sickness among group members, project supervisor or Mobai. 
&
The team shall follow the regional and local Covid-19 measures to the greatest extent. Members becoming ill should be kept up to date about the progress of the project. Additionally, to prevent rescheduling, members should know what the rest of the group is working on. Daily meetings will be an effective countermeasure. 
\\ \hline
\cellcolor[HTML]{F8FF00}R1  &
Group members leaving the project. 
&  
A structured group with good planning and communication reduces the risk of members leaving the project. A safe team atmosphere where members are encouraged to bring fourth new ideas and input is essential. The team members should also be given the opportunity to work with something they find interesting and fun. Should however a member leave the team, a redistribution of the remaining work tasks shall be equally split amongst the remaining members. 
\\ \hline
\cellcolor[HTML]{32CB00}R5  &
A similar product launches. 
&
If a similar product launches, it is important to meet all of Mobai´s specialized needs and avoid delays in the deployment. 
\\ \hline
\cellcolor[HTML]{32CB00}R14 &
Unrealistic demands are made along the way by the Product Owner. 
&
Our agile development method will mitigate large reschedules of the project if new demands are provided. However, our main task is written in the task description (\ref{sec:TaskD}) and any additional demands are not required by us to complete. 
\\ \hline
\end{tabular}%
}
\end{table}



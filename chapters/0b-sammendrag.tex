\chapter*{Sammendrag}
Ytelsen på ansiktgjenkjenningssystmer er avhengig av kvaliteten på bildene for å kunne teste og trene disse systemene. For å automatisk vurdere kvaliteten på ansiktsbilder blir Face Image Quality Metrics (FIQMs) benyttet. Slike kvalitetsmetrikker gir objektive resultater som korresponderer med hvor synlig ansiktet i et bilde er. I dette prosjektet introduserer vi en webapplikasjon som regner ut slike objektive resultater med to moderne FIQMs. For å vurdere nøyaktigheten på disse metrikkene, samlet vi inn subjektive data fra eksperter og ikke-eksperter på tre forskjellige dataset. Vi samlet også inn et nytt dataset med ansiktsbilder som vi mener er overlegent i forhold til andre nåværende dataset. Denne overlegenheten er med tanke på antall bilder, type forvregninger bildene er påvirket av og hvordan bildet er tatt (bilder med ansiktsmasker og bilder med skrå vinkler). Resultatene viser at de objektive resultatene regnet ut av de to kvalitetsmetrikkene har lav korrelasjon med de subjektive resultatene samlet inn via subjektive eksperimenter. 

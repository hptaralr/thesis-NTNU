\chapter{Conclusion}
\label{chap:Conclusion}
 Now that we have looked at the various aspects that went into the development of the web application and the subjective experiment, it is time to reflect on the journey our group has gone through the last months. In Section \ref{sec:objective} we discuss how the application turned out and how the user test supported that the requirements were fulfilled. Further in Section \ref{sec:subjectiveevaluation}, we reflect upon our subjective experiments. In Section \ref{sec:resultsevaluation} we summarize our results. To end the chapter, we evaluate the group work (Section \ref{sec:groupworkevaluation}), learning outcome (Section \ref{sec:learningoutcome}) and discuss future work (Section \ref{sec:futurework}).

\section{Objective Assessment Evaluation}
\label{sec:objective}
To evaluate our work done in the objective assessment, we have looked at how well we managed to complete the task Mobai gave us. Based on the requirements stated in Section \ref{sec:requirements} and the use cases showcased in Figure \ref{fig:UseCase}, we can see that we have managed to build and complete all the main functionality and requirements that Mobai had given us in the objective assessment. Given the feedback from the user testing showcased in Table \ref{table:usertesttasks}, all functionality worked and was set to ``Satisfactory'' by the Product Owner. From the user testing questions in Table \ref{table:usertestquestions}, we got some feedback regarding doing some minor design changes to the frontend. Overall we have done what Mobai asked us and according to their feedback, they were satisfied. 

\section{Subjective Experiment Evaluation}
\label{sec:subjectiveevaluation}
As introduced in Section \ref{sec:SubjectiveAspects}, there were several aspects to take into account when conducting a subjective experiment. We spent much time researching and analyzing different aspects and platforms when deciding the structure of the experiment. Collecting observers was more difficult with the ongoing COVID-19 restrictions. Ideally we wanted to invite observers to perform the subjective experiment in a controlled environment which was not possible. However, conducting a subjective experiment on the web ended up being surprisingly less complicated than we thought. 

\section{Evaluation of the Results}
\label{sec:resultsevaluation}
When it comes to the two FIQMs, ISO Metrics and FaceQnet, we can conclude that they performed similar on the Combined passport alike dataset with a low to moderate correlation shy of 0.5 with the subjective scores, as shown in Figure \ref{fig:corrISOsvsSub} and Figure \ref{fig:corrFACEQNETsvsSub}. On the Capture from photo dataset, ISO Metrics outperformed FaceQnet by having a low to moderate correlation, while FaceQnet´s correlation coefficients were considered weak. The dataset in which the FIQMs performed crucially poor, were the Selfie dataset. Both FIQMs had a non-existent association with the subjective scores. With regards to FaceQnet, the FIQM did not work for that particular dataset, because the faces in the images were not detected and therefore the cropping phase became defective. ISO Metrics on the other hand, struggled with estimating the correct quality because the facial images of the Selfie dataset greatly differed in quality relative to the two other datasets. The facial images were either off-centered, too zoomed in or a combination of both, which ISO Metrics did not react accordingly to. The strongest association achieved was done by using a weighted average of the scores from ISO Metrics and FaceQnet on Combined passport alike. With Pearson and Spearman values around 0.6, we can conclude that using a weighted average approach of the FIQMs can be a reliable choice for datasets where the facial images checks several of the bullet points in what defines a good facial image listed in Section \ref{sec:setup}.

The performance of the FIQMs with regards to face masks were subpar. The perceived quality by ISO Metrics did not change with different face masks coverings, whereas FaceQnet had very minor differences. The two FIQMs can not be considered reliable when assessing face masks. The same can not be said for oblique angled facial images. The FIQMs showed minor differences when assessing those types of facial images. 

Assessing the quality of distorted facial images did affect the perceived quality for ISO Metrics, but not for FaceQnet. The last mentioned FIQM perceived the quality consistently with close to perfect association with the original images shown in Figure \ref{fig:HistogramFACE} and Figure \ref{fig:NFCFaceCorr}. The FIQM should be considered accurate when predicting the perceived quality on the distortions we added, which was not entirely the case for ISO Metrics. The FIQM had trouble assessing facial images with noise and was less accurate than FaceQnet when assessing blur and compression done in Photoshop shown in Figure \ref{fig:NFCISOCorr}. 

\section{Group Work Evaluation}
\label{sec:groupworkevaluation}
Although only working from home and communicating on digital platforms, the collaboration between the group members has been great. We have mostly worked together and all agreements and deadlines were met. The workflow throughout the project was consistent which ended up in great quality work. Toggl (Appendix \ref{app:toggl-reports}) was a great tool for following each members working tasks. The group members have helped each other with reviewing their work. The thesis writing started early and we are satisfied with the outcome. 

\section{Learning Outcome}
\label{sec:learningoutcome}
By working closely together the past months, we have learned the importance of having a strict schedule and expectations among the members to ensure everyone keep their morale and the good work throughout the project. We experienced that not having clear deadlines resulted in incompletion of working tasks. These types of lacks required more time spending on reviewing the code and report. On the programming side, the group has learned new programming languages, tools, libraries and how these interact. We may not have come up with the best approach, however the requirements were met and the web application is working. No one in the team had developed a web application before, resulting in a new experience for all members. Coding wise, it was manageable to create the backend in Python and a frontend UI in React. However, no one had experience in making them interact with each other, so the APIs could be developed in a more well-designed way. In terms of research, the members have developed new abilities throughout the project by retrieving information and using it in our work. This project has also improved our English and how to present our work in a more scientific way.  

\section{Future Work}
\label{sec:futurework}
We managed to include everything specified in Chapter \ref{chap:objective}. Since one of the requirements was to easy facilitate new FIQMs to be added into the web application, using the application in further work was a reality. Mobai has stated that this application will be a part of their face recognition system, where the web application works as a sieve to filter out low-quality images. Chapter \ref{chap:objective} can be used as documentation to describe the different technologies used to develop the web application.

The two FIQMs utilized in this project, can be studied further. Whereas both provide quality scores to facial images, they act differently. To give a better evaluation of facial images, studying the weighted average between the two FIQMs can provide improved accuracy of the quality scores. Proposing a new weighted approach where the FIQMs are weighted differently may be the next step in achieving improved performance of the metrics.

Our dataset, described in Section \ref{sec:ownData}, was outperforming the datasets provided by Mobai. It consisted of more facial images, distortions and new aspects. The new aspects consisting of camera angels and wearing face masks tested the FIQMs in a new way. Mobai is able to use the NFC dataset in their further work of testing new FIQMs and face recognition systems. 

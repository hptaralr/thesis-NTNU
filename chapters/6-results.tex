\chapter{Results and Discussion}
\label{chap:Results}
Since a large part of our bachelor task consisted of face image quality research, this chapter is of great importance. After conducting the experiments in Chapter \ref{chap:subjective}, we gathered large amount of data, which we had to process. This data will be presented in the form of statistical calculations, such as box plots, correlation graphs and standard deviation graphs. The main goal of the chapter is to tie together the results we achieved in the objective and subjective face quality assessment, discuss them and present an in-depth look into the performance of the FIQMs relative to our collected ground truth data.   

\section{Main experiment}
The results we achieved are split into two sections. In this first section the results of the objective assessment and our first subjective experiment will be presented and discussed. Both assessment categories are based on the three datasets provided by Mobai, which were described in Chapter \ref{chap:subjective}. 
%Utfylle mer her senere om hvordan seksjonen er oppbygd.

\subsection{Objective assessment scores}
The FIQMs presented in Chapter \ref{chap:FQA} use different approaches to predict the perceived face quality. Their predicted perception of quality is supposed to correlate with human assessment, which is the whole purpose of the metrics. The objective predictions produced by the FIQMs were measured up to human assessment to evaluate how true to their purpose they were. This evaluation was carried out by comparing the results of the FIQMs against the results gathered from human observers. The same resized images used in our first subjective experiment were used as input to our two FIQMs. 



\subsection{Subjective assessment scores}

\section{Second experiment}
This second section is about the second subjective experiment we conducted based on our own dataset of 250 images. 
\subsection{Objective assessment scores}
\subsection{Subjective assessment scores}
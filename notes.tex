- Abstract (NO/ENG)
- Contents
- Figures
- Tables
- Code Listings
- Acronyms (legg til i glossary.tex) HVORDAN VISE I F DOKUMENTET?!?
- Acknowledgements

1. Introduction (6-8)
 - Background: about Mobai (1/2)
 - Scope: subject area, delimitation, task description (1-2)
 - target groups (1/4)
    * users of the web application
    * Thesis reader
 - Why did we choose this task? (1)
 - The team (1-2)
   - Who we are / the team members (1/2)
   - Academic backgrounds (1/2)
   - Roles (1/2)
   - Decision making
 - Structure of the report (1)
 
 2. Specification (4-5)
    - Choice of development method (2-3) 
        - considered models (1)
    - Risk analysis
    
    
3. Face Quality Assessment (6-7)
    - What is a good image (Quality) (2-3)
    - About Face quality metrics (3)
        - ISO (1)
        - FaceQNet (1)
  
  
4. Objective assessment (15-20)
    - Application requirements (4-5)
        * Use cases (3)
        * Misuse cases (1-2)
    - Why have we chosen web (react) + python (flask)? (1-2). docker(containerization) 
    - explain frontend (react, material-ui) (1)
    - explain backend (axios) (restAPI) (1)
    - explain why we use containerization (docker) (1)
    - Implementation (2)
        * How they all work together
    - Testing: unit testing, user testing (1-2)
    - Results (1)
    

5. Subjective experiment (16-20)
   - Conducting the survey  (6-7)
        * Factors (1) 
        * Our choice (1-2) 
        * Corona disadvantages (1/2)
        * Introduction manual (2) 
        * Type of people/participants (1/2)
        * What went wrong (1)
    - Datasets (6-9)
        * Combined passport alike (1)
            - Cameras used
        * Capture from photo ntnu dataset (1/2)
        * Selfie dataset (1/2)
        * Combining datasets (2)
        * Creating our own dataset (3-4)
            - Masker/ munnbind 
            - Tilting 
    - Results(3) 
        * Overall (3/2)
        * Mobai (1/2)
        * Normal people (non-experts)(1/2)
        * Correlation between Mobai and normal people(1)
            - The tools
 
6. Results (3)
     - discuss correlation between subj/obj. assessment (3)
  
7. Conclusion (4)
    - can the two algorithms be used further? (2)
    - how are the objective similar to the subjective? same quality scores? (2)
  
 Appendix
    - Project plan
    - Survey instructions
    - Meeting reports with supervisor
    - Meeting reports with Mobai
    - Results (raw data)
    
 
 ----
 Alltid skrive "FIQMs"? Kan vi også bruke "metrics"?
 Python with p or P?
 Mellomrom mellom tekst og citation
 Parenteser rundt figurnavn eller ikke?
 
 Photoshop: 
 Compression: Level 1
 Blur: 2
 Noise: 7 \%
 
%The three image lineups in Figure \ref{fig:ICAOlineup} showcases typical flaws of facial images that negatively affect the quality. The last image of each lineup is considered to have the best quality. The flaws described varies in severity. The face image quality of the lady with her face covered would be at the bottom of the quality scale. The first two images of the man, would not affect the quality to the same degree and the quality of the first two images of the kid, would affect the quality even less. 
 
 
                                       /;    ;\
                                  __  \\____//
                                 /{_\_/   `'\____
                                 \___   (o)  (o  }
      _____________________________/          :--'
  ,-,'`@@@@@@@@       @@@@@@         \_    `__\
 ;:(  @@@@@@@@@        @@@             \___(o'o)
 :: )  @@@@          @@@@@@        ,'@@(  `===='
 :: : @@@@@:          @@@@         `@@@:
 :: \  @@@@@:       @@@@@@@)    (  '@@@'
 ;; /\      /`,    @@@@@@@@@\   :@@@@@)
 ::/  )    {_----------------:  :~`,~~;
;;'`; :   )                  :  / `; ;
;;;; : :   ;                  :  ;  ; :
`'`' / :  :                   :  :  : :
   )_ \__;      ";"          :_ ;  \_\       `,','
   :__\  \    * `,'*         \  \  :  \   *  8`;'*
       `^'     \ :/           `^'  `-^-'   \v/ :
       
Julian er kuuuul